\chapter{相关研究进展}
\section{引言}
自从计算机技术应用到医学影像分析以来,有许许多多医学影像分析难题因其具有的重大临床应用价值和实际意义而引起研究者的浓厚兴趣并为之投入大量时间和精力,生物标记物的精确定位便是其中之一。本章将会生物标记物精确定位任务的常用数据集进行介绍。接着将会介绍本文方法中涉及到的最为重要的基本知识点,包括卷积神经网络、对抗生成网络和自编码-解码器。
下一步,我们将对可用于完成生物标记物精确定位任务的方法进行详细(包括传统的多示例学习和当下流行的卷积神经网络方法)介绍,力求将相关方法阐述得清晰明了,突出比较各种方法在生物标记物的精确定位任务上的长处和不足。在本章最后,我们将给出实验结果的评判标准,并阐述选择这些比较标准的合理性。

\section{常用数据集}
一旦展开对选取问题的具体研究,第一步便是选取实验所使用的数据集。尤其对于当下十分火热、有着数据驱动特性的卷积神经网络来说,选择一个合适的数据集的重要性更加不言而喻。再加上医学影像数据获取、标注成本更高,导致可用于生物标记物的精确定位测试的公开数据集较少。由于医学数据很可能会涉及到病人的隐私,为了尽可能保护病人权益,防止病人信息泄露,图像数量较多的、图像质量较高的数据集往往存于知名公立医院且未公开。在已然公开的数据集中,本文将选取一些知名度较高的、在学术界被研究者广泛接受的、图像质量相对较高的数据集进行介绍。

\subsection{眼底病变数据集}
眼科学是临床医学的一个独特分支。眼科的影像学检查方法有眼底摄影、光学相干断层扫描、眼底荧光血管造影、扫描激光检眼镜等。一个明确的眼科疾病诊断需要结合几个不同的试验结果。在临床实践中,诊断和治疗策略的确定依赖于影像学资料的评价。目前眼底照片已广泛应用于青光眼和视网膜疾病等眼科疾病的诊断。然而,眼成像数据的解释需要大量的经验和时间。在这里,我们介绍部分具有良好注释和标签的数据集,扼要信息如表~\ref{tab:datasets_info}所示,下面分别进行简单介绍。

注意,图像级标注指的是将图像标注为对应类别,如正常/异常。像素级标注在图像级标注基础上还标出病变的准确位置。另外,图像数量指的是数据集中官方有提供标注的图像数据。以上说明同样适用于表~\ref{tab:skin_datasets_info}。

\begin{table}[h]
	\centering
	\caption{常用眼底病变数据集。}		
	\label{tab:datasets_info}
	\begin{tabular}{c|c|c|c}
		\toprule[2pt]
		数据集名称 & 图像数量 & 类别 & 标注 \\
		\midrule[2pt]
		Kaggle Diabetic Retinopathy (DR)	& 35,127	& 5	&图像级 \\
		\hline                         
		iChallenge Glaucomatous Optic Neuropathy (GON)    & 1,200    & 2 & 图像级 \\ \hline
		iChallenge Age-related Macular Degeneration (AMD) & 1,200    & 2 & 部分像素级 \\ \hline
		iChallenge Pathological Myopia (PM)               & 1,200    & 2 & 部分像素级 \\ \hline
		ODIR-5K & 7,000 & 8 & 图像级 \\ \hline
		
		Indian Diabetic Retinopathy Image Dataset (IDRiD) & 516 & 5 & 部分像素级 \\
		\bottomrule[2pt]
	\end{tabular}
\end{table}

糖尿病视网膜病变是发达国家劳动年龄人口失明的主要原因。DR数据集\footnote{https://www.kaggle.com/c/diabetic-retinopathy-detection/data}是目前关于糖尿病视网膜病变的最大数据集,提供了在各种成像条件下拍摄的高分辨率视网膜图像。目前在Kaggle上开源数据中,训练集有35,127张样本,每张图像尺寸均大于$1000\times 1000$但大小不等,目前只有图像集标注。专业医师根据患者患病程度将每张图像标注为0至4共5类。0、1、2、3和4分别代表未患糖尿病视网膜病变、轻微糖尿病视网膜病变、中度糖尿病视网膜病变、严重糖尿病视网膜病变和增生性糖尿病视网膜病变。标注数字越大代表患病越严重。

GON数据集\footnote{http://ai.baidu.com/broad/subordinate?dataset=gno}是关于青光眼眼底照片的数据集,共包含1,200张彩色眼底照片。并平均分为训练集、验证集和测试集。其中,训练集图像由德国蔡司眼底照相机拍摄,尺寸大小为$2124\times 2056$,验证集和测试集图像由佳能眼底照相机拍摄,尺寸大小为$1634\times 1634 $。所有图像均是图像集标注,标记为青光眼/非青光眼,均以后极为中心,伴有黄斑和视盘。

AMD数据集\footnote{http://ai.baidu.com/broad/subordinate?dataset=amd}是关于年龄相关性黄斑变性眼底照片数据库,共有1,200张彩色眼底照片可供选择。这些照片来自非AMD受试者(约77\%)和AMD患者(约23\%)。提供AMD/非AMD的标签,椎间盘边界和中央凹的位置,以及各种病变的边界,以训练模型进行自动AMD评估。数据集中每个样本都有图像级标注,只有部分样本有像素级标注,标注了与年龄相关性黄斑变性相关的四种典型异常。

近视已成为全球公共卫生的负担。为了促进近视的研究,PM数据集\footnote{http://ai.baidu.com/broad/subordinate?dataset=pm}是病理性近视眼底照片数据库,提供了1200个来自非病理性近视受试者和病理性近视患者(约50\%)的标注视网膜眼底图像的大数据集。每个图像样本同样均有图像级标注,部分图像有包括斑片状视网膜萎缩(包括乳头周围萎缩)和视网膜脱离在内的两种典型异常的像素级标注。

ODIR-5K数据集\footnote{https://odir2019.grand-challenge.org/dataset/}是一个结构化的眼科数据库,其中包括5,000名患有年龄的患者,双眼的彩色眼底照片和医生的诊断关键词。注意只有3,500名患者(7,000张样本)数据作为训练集,并且有图像级标签。该数据集是上工医疗技术有限公司从中国不同医院/医疗中心收集的“真实”患者信息。专业医师将患者分为8个标签,包括正常,糖尿病,青光眼,白内障,年龄相关性黄斑变性,高血压,近视和其他疾病/异常。由于存在部分病人同时患有多种疾病,因而部分图像有多个标签。

IDRiD\footnote{https://idrid.grand-challenge.org/Data/}眼底图像是由印度一家眼科诊所的视网膜专家收集的。数据集共包括516张样本,均提供了典型糖尿病视网膜病变病变和正常视网膜结构的专家标记。数据集所有图像都集中在黄斑附近。图像分辨率为$4288\times 2848$像素,存储为jpg文件格式。此外,它还根据国际临床相关性标准,为数据库中的每张图像提供关于糖尿病视网膜病变的疾病严重程度和糖尿病黄斑水肿的信息。与DR数据集一样,它一共将图像分为5类。与DR数据集不同的是,IDRiD数据集有81张患病彩色眼袋图像有精确像素级标注,如微动脉瘤、软渗出物、硬渗出物和出血。

眼底病变往往有病变区域较小,病变区域数量较多,病变区域分布较分散的特点。因此,眼底有些病变区域容易混淆,比较难发现。另外,眼底图像较为精细,各种细节纹理丰富,故发现眼底病变的生物标记物通常极具挑战性。

\subsection{黑色素瘤皮肤病变图像}
黑色素瘤是多种皮肤癌中最致命的一种。黑色素瘤是发生在皮肤表面的色素性病变,可以通过专业医师的视觉检查早期发现。黑色素瘤也适用于自动检测与图像分析。皮肤镜检查是一种皮肤成像方法,与无辅助的视觉检查相比,已证明可改善皮肤癌的诊断。为了更广泛地提供专业知识,国际皮肤成像协作组织开发了专门档案,这是一个国际皮肤镜图像库,可用于皮肤科专业医师的临床培训,也能用于举办比赛,寻求计算机算法解决临床问题。在这里,我们介绍三个图像质量较高,可用于生物标记物定位的黑色素瘤病变数据集。数据集名称、图像数量等基本信息如表~\ref{tab:skin_datasets_info}所示。


\begin{table}[h]
	\centering
	\caption{常用眼底病变数据集。}		
	\label{tab:skin_datasets_info}
	\begin{tabular}{c|c|c|c}
		\toprule[2pt]
		数据集名称 & 图像数量 & 类别 & 标注 \\
		\midrule[2pt]
		
		International Skin Imaging Collaboration (ISIC) 2017 &  $\sim 2,300$ & 3 & 部分像素级 \\ \hline
		International Skin Imaging Collaboration (ISIC) 2018 & $\geq 12,500$ & 7  & 部分像素级 \\ \hline
		International Skin Imaging Collaboration (ISIC) 2019 & 25,331 & 8  & 图像级  \\ 
		\bottomrule[2pt]
	\end{tabular}
\end{table}

ISIC2017数据集~\cite{codella2018skin}中大约有2,300张皮肤镜图像,其中大约2,150张图像是训练集,剩下约150张图像是验证集。图像尺寸大小在$400\sim 600$之间。数据集包括黑色素瘤、脂溢性角化病和良性的痣(可看做正常)在内的3种类别。

ISIC2018数据集~\cite{codella2019skin, tschandl2018ham10000}中有超过12,500张皮肤镜图像。图像尺寸大小在$400\sim 600$之间。包括光化性角化病(日光性角化病)和上皮内癌(鲍文氏病)、基底细胞癌、良性的角化病、皮肤纤维瘤、黑素细胞痣、黑素瘤、血管皮肤损伤共7类。

ISIC2019数据集\footnote{https://challenge2019.isic-archive.com/}中有25,331张皮肤镜图像,包括黑色素瘤黑素细胞痣、基底细胞癌、光化性角化病、良性角化病(太阳扁豆/脂溢性角化病/扁平苔藓样角化病)、皮肤纤维瘤、血管病变、鳞状细胞癌和没有以上病变类型在内的8个类别。图像尺寸大小在$400\sim 600$之间。

与眼底病变类型不同的是,黑色素瘤的各种病变类型往往在皮肤镜上所占区域比较大,通常所占比例在$1/3$以上。各种病变类型之间的区别主要表现在病变区域细微纹理区间上。

\section{基本知识要点}
本小节主要是介绍生物标记物定位任务的概念和目前常用的模型方法。首先,我们会介绍生物标记物的概念,让读者对本文研究主题有清晰的认识。接着会介绍卷积神经网络、自动编码器-解码器等重要基本要点。下一步介绍目前已有的常用解决模型方法。最后介绍对各种模型方法性能比较的评价标准。
\subsection{卷积神经网络}
卷积神经网络,
\subsection{自动编码器-解码器}
\subsection{对抗生成网络}
%	\section{生物标记物定位方法}
%\subsection{多示例学习}
%\subsection{卷积神经网络}
%\section{评价标准}
%\section{本章小结}