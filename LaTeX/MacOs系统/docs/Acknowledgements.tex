\chapter{致谢}

%由衷感谢我的导师某某教授,本文是在他的指导下完成的。……

%(谢辞应以简短的文字对课题研究与论文撰写过程中曾直接给予帮助的人员(例如指导教师、答疑教师及其他人员)表示对自己的谢意,这不仅是一种礼貌,也是对他人劳动的尊重,是治学者应当遵循的学术规范。)
时光飞逝,转眼之间三年令人难忘也令人有无限追忆的研究生生涯即将结束。三年前的自己对医学图像处理毫不了解,只是知道计算机视觉领域的零星两三名词概念,对于科研工作和科研生活,更是一无所知。转眼之间,三年光阴如白驹过隙,转瞬即逝。每天两点一线的生活轨迹,经常遇到的各种代码BUG,绞尽脑汁的论文写作,还有那些令人沮丧的实验结果,这些看似枯燥而又单调的事情,经历之后才知道正是这些让我不再浮躁,让我变得更加严谨,也训练了我分析问题、解决问题的科学方法与良好习惯。在实验室挑灯夜战、只有青灯黄卷相伴的日子里,我慢慢叩开了科研工作的大门。对于这些日子,我至今记忆犹新,难以忘怀,也给了我今后面对困难的勇气和客服困难的底气,纵使困难重重,亦不退缩。在此,对于那些帮助过我的人,我致以最真挚的感谢,祝福你们在今后的生活里一切顺利。

首先,我要感谢我的导师王瑞轩教授。韩愈在《师说》里写道,师者,所以传道受业解惑也。我始终认为我的导师是将这句话奉为圭臬的。一方面,恩师从事医学影像分析相关科研工作多年,有着渊博的学识和丰富的科研经验。另一方面,恩师有着严谨的治学态度和勤勉的工作态度,以身作则,在潜移默化中不断影响着我,恩师的敦敦教诲,至今不时在我的脑海中浮现。记得在2019年春季开学之际,我的一篇论文投稿被拒,再加上毕业要求、暑期实习的压力,我产生了退缩心理。正是在恩师的耐心鼓励和冷静分析下,我重拾信心,再次在当年4月份投稿医学影像分析顶会,幸运的是,这次我中稿了。在这里,我再次对我的导师致以最真诚的谢意。

然后我要感谢指导过我的郑伟师教授。感谢郑老师对我的关怀,非常支持科研工作,给我提供了很多了便利。在赶论文期间,除了帮助我修改论文、提供宝贵的修改意见外,还给我提供充足的计算资源。

除了要感谢曾经指导过我的这些老师外,我还要感谢实验室给我提供过帮助的同学。比如和我多次讨论问题的陈泽林同学、在投稿MICCAI 2019期间和我一起合作的谭书涵师弟、一起合作投稿ISBI 2020的雷伟贤师弟。在这三年里,正是有了你们的帮助与陪伴,我在科研路上才不会孤单。

在此毕业临别之际,言语已无法表达出我心里对你们的谢意,请容我对你们道一声谢谢,希望在以后的日子有缘再次合作。


% \vskip 108pt
% \begin{flushright}
% 	张荣\makebox[1cm]{} \\
% 	\today
% \end{flushright}

