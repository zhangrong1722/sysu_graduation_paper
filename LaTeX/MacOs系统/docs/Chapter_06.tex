\chapter{总结与展望}
%本章是毕业论文的总结,是整篇论文的归宿,应精炼、准确、完整。应着重阐述自己的创造性成果及其在本研究领域中的意义、作用,还可进一步提出需要讨论的问题和建议。
\section{工作总结}
近些年来,随着计算机技术的蓬勃发展及其在医学影像领域的广泛应用,生物标记物定位方法也不断取得进步。由于医学图像的标注代价高昂、难以大量获得,故在弱监督条件(提供图像级标签,需要给出像素级定位结果)下实现生物标记物的定位就具有较大的研究价值和实际意义。针对弱监督条件下的生物标记物定位问题,目前主要有包括多示例学习和卷积神经网络的可视化在内的两种计算机方法可解决生物标记物定位这一问题。但是在面对分布广泛、尺寸大小不一的生物标记物时,以上方法均只能定位到生物标记物的大概位置,同时还会伴随着将正常区域也误认为生物标记物和遗漏生物标记物的的情况发生,无法在弱监督条件下实现生物标记物的精确定位。为了克服以上方法的缺陷,在弱监督条件下实现生物标记物的精确定位,本文从图像生成的角度,创造性地组合了CNN分类器和判别器,提出了一种新型网络结构。
\section{研究展望}
