%%
% 摘要信息
% 摘要内容应概括地反映出本论文的主要内容,主要说明本论文的研究目的、内容、方法、成果和结论。要突出本论文的创造性成果或新见解,不要与引言相 混淆。语言力求精练、准确,以 300—500 字为宜。
% 关键词是供检索用的主题词条,应采用能覆盖论文主要内容的通用技术词条(参照相应的技术术语 标准)。按词条的外延层次排列(外延大的排在前面)。


\cabstract{
	近些年来,随着深度学习的高速发展以及广泛应用,生物标记物定位算法无论是在精确性还是在应用广泛性上都取得了喜人的进步。在临床情况下,对于专业医师,标出生物标记物的大概位置(例如,矩形边界框)通常相对容易,但是标出生物标记物的精确位置(像素级)对于专业医生是非常困难甚至不可能的,尤其在生物标记物在医学图像中分布广泛,大小各异,边界模糊的情况下。因此,借助深度学习手段去完成生物标记物发现任务非常有必要。另外,医学图像领域中,像素级别图像标注(如医学分割任务)获取代价高昂,不仅需要大量经验丰富的专业医师,而且数据获取-数据标注周期较长,对于某些特定疾病的标注还十分困难(例如糖尿病型视网膜病变)。幸运的是,图像级别的图像标注(类别标签)相对简单,获取较为容易。因而在弱监督条件下(提供图像级别标签,给出像素级别结果)完成生物标记物精确定位任务也显得非常有必要。鉴于以上情况,本文提出一种组合卷积神经网络和对抗生成网络的新型网络结构来完成弱监督条件下的生物标记物精确定位任务,本文主要贡献如下:
	
	1)到目前为止,在弱监督条件下,还没有相关算法直接完成生物标记物精确定位任务,本文创新性将卷积神经网络和对抗生成网络组合为一种新的深度卷积神经网络,其中对抗生成网络由生成器和判别器组成,卷积神经网络充当分类器角色。为了给出像素级定位结果,从而实现生物标记物的精确定位,本文从图像生成的角度,输入一张异常图像,将生成器的输出减去输入。在深度卷积神经网络训练方面,采用生成器-分类器和生成器-判别器两步交替训练策略。
	
	2)不仅将本文提出的网络结构用于处理二类问题,还将其推广到处理多类问题。
	本文提出的网络结构在糖尿病型视网膜病变数据集和人造生物标记物数据集上均取得了目前最领先结果。一方面填补了在弱监督条件下直接完成生物标记物精确定位任务空白,另外一方面,相较于其他间接算法通常借助可视化手段对特征图上采样得到热图,将神经网络关注区域视为生物标记物,这种方式无法实现生物标记物精确定位,本文提出的算法则可以通过生成器去除生物标记物,所以生成器输出减去输入即可得到生物标记物的精确位置。本文提出的算法还无须对生物标记物的精确标注,只需要类别标注,就可以得到对应疾病的生物标记物,免去了专业医师标注的标注成本。
	
}
% 中文关键词(每个关键词之间用“;”分开,最后一个关键词不打标点符号。)
\ckeywords{生物标记物定位;编码器-解码器;对抗生成网络;弱监督 }

\eabstract{
	{\color{red} TODO in the final writing.}
}
% 英文文关键词(关键词之间用逗号隔开,最后一个关键词不打标点符号。)
\ekeywords{Biomarker Localization, Encoder-Decoder, Generative Adversarial Networks, Weak supervision}